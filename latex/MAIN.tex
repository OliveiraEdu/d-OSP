\documentclass[final]{rc-book-2.14}
\usepackage[utf8]{inputenc}
\usepackage[T1]{fontenc}
\usepackage{rc-natbib-2.14}
\usepackage{rc-tabular-2.14}
\usepackage{rc-listing-2.14}

\loadlanguages{english,portuguese}
\settypewriterfont{\sffamily\smaller}

% Dirty-Tricks.tex

\newcolumntype{S}{>{\centering\arraybackslash}p{1.7cm}}
\newcolumntype{T}{p{6cm}}




\includeonly{Abstract, MAIN}

\begin{document}

% Cover.tex

% Title page

\title{Aplicações Decentralizadas}
\subtitle{Uma abordagem para a Melhoria de Experimentos Pré-clínicos}

\author{Eduardo Costa de Oliveira}
\institution{Universidade Regional do Noroeste do Estado do Rio Grande do Sul}

\documenttype{Dissertação de Mestrado}

\advisor{\hfill \linebreak \linebreak Orientador: \linebreak Dr.~Rafael Zancan Frantz \linebreak Coorientador: \linebreak Dr.~Thiago Heck}

\splash{
    \mbox{}\hfill%
    \psfigure[width=5.294192cm,height=4.913207cm]{fig/logo-unijui}%
    \hfill%
    \psfigure[width=5.294192cm,height=4.913207cm]{fig/logo-gca}%
    \hfill\mbox{}
}

\date {Março, 2023}

% Indexing page

\publisher{
    First published in March 2023 by \\
    Applied Computing Research Group - GCA \\
    Department of Exact Sciences and Engineering \\
    Rua Lulu Ilgenfritz, 480 - São Geraldo \\
    Ijuí, 98700-000, Brazil.
}

\copytext{
    Copyright \copyright\ \toroman{2012} Applied Computing Research Group \\
    \url{http://www.gca.unijui.edu.br} \\
    \email{gca@unijui.edu.br}
}

\rights{\defaultrights}

%\record {
%    D.1.2 [Automatic Programming];
%    D.1.3 [Concurrent Programming];
%    D.2.6 [Programming Environments]: Integrated environment, Programmer workbench;
%    D.2.10 [Design]: Representation;
%}

\support{
    Bolsa de mestrado concedida pela [FAPERGS/CAPES]. A pesquisa desenvolvida neste trabalho também teve o apoio dos seguintes projetos Grupo de Pesquisa em Computação Aplicada: [listar-códigos-dos-projetos].
}

% Minutes page

\minutestext{\defaultminutestext{Mestre}{Modelagem Matemática}}

\minutesdate{\defaultminutesdate}

\boardmember{
     Dr. Rafael Zancan Frantz\\
     UNIJUÍ\\
     (Orientador)
}

\boardmember{
     Dra. Thiago Heck\\
     UNIJUÍ\\
     (Co-orientadora)
}

\boardmember{
     Dr. Rafael Corchuelo\\
     Universidade de Sevilha
}

\boardmember{
    Dr. Sandro Sawicki \\
    UNIJUÍ
}




% Art page

\artwork{
    \psfigure{fig/eai-by-pedrinho.eps} \\
    \bigbreak\bigbreak\bigbreak\bigbreak    
    {Integração de Aplicações por Pedrinho, 7 anos de idade.} \\
}

% Dedicatory page

\dedicatory{Dedico este trabalho a ...
}



\makefront 
    
%====================================================================
\chapter{Índice de abreviaturas}
\label{app-acronym}
%====================================================================

\begin{description}
    \item[EAI -] Integração de Aplicações Empresariais (\textit{Enterprise Application Integration})
    \item[DSL -] Linguagem de Domínio Específico (\textit{Domain Specific Language})
    \item[API -] Interface de Programação de Aplicativo (\textit{Application Programming Interface})
\end{description}
    

%====================================================================  
\chapter{Índice de símbolos}
\label{app-acronym}
%====================================================================

\begin{description}
	\item[ \textit{n} -] parâmetro que representa o número de threads.
	\item[ $\lambda$ -] taxa de chegada de mensagens na solução.
\end{description}
    

%==================================================================== 
\chapter{Agradecimentos}
\label{chp:acknowledgements}
%====================================================================

\drop Agradeço primeiramente a ....
    


%====================================================================
\chapter{Resumo}
\label{chp:general-abstract:portuguese}
%====================================================================

\begin{quotation}[Filósofo Romano (106 AC - 43 AC)]{Marcus T.~Cicero}
    O início de todas as coisas é pequeno.
\end{quotation}


\begin{otherlanguage}{portuguese}
    \drop Escreva aqui um resumo para a sua dissertação.
\end{otherlanguage}


%====================================================================
\chapter{Abstract}
\label{chp:general-abstract:english}

%====================================================================

\begin{quotation}[Roman philosopher (106 BC - 43 BC)]{Marcus T.~Cicero}
    The beginnings of all things are small.
\end{quotation}

\drop Write here an abstract for your dissertation.

\mainmatter


%====================================================================
\chapter{Introdução}
\label{chp:introduction}
%====================================================================

\begin{quotation}[]{Chinese proverb}
    The journey of a thousand miles \\ begins with a single step.
\end{quotation}


%-------------------------------------------------------------
\section{Contexto da Pesquisa}
\label{sec:introduction:context}
%-------------------------------------------------------------

Aqui você deve contextualizar seu trabalho, utilizando uma abordagem descritiva topdown, isto é, iniciar introduzindo a área na qual o trabalho é desenvolvido até chegar no tema central tratado na dissertação. Destacar a importância da área e, portanto, do tema abordado nessa área.

 
%-------------------------------------------------------------
\section{Motivação}
\label{sec:introduction:motivation}
%-------------------------------------------------------------

Nessa seção você deve descrever qual é o problema que motiva o desenvolvimento dessa dissertação e porquê isso é um problema. Essa seção deve deixar clara a motivação da sua pesquisa perante o problema. No final da motivação, em um parágrafo a parte, sugere-se formular a hipótese geral da pesquisa. 
 

%-------------------------------------------------------------
\section{Objetivos}
\label{sec:introduction:goal}
%-------------------------------------------------------------

Breve parágrafo de introdução dessa seção.


%-------------------------------------------------------------
\subsection{Geral}
\label{subsec:introduction:goal:general}
%-------------------------------------------------------------

\begin{citeverbatim}
Descreva em um breve parágrafo qual é o objetivo principal de sua pesquisa nessa dissertação.
\end{citeverbatim}


%-------------------------------------------------------------
\subsection{Específicos}
\label{subsec:introduction:goal:specific}
%-------------------------------------------------------------

\begin{itemize}
  \item Objetivo específico 1
  \item Objetivo específico 2
  \item Objetivo específico 3
  \item Objetivo específico 4
\end{itemize}


%-------------------------------------------------------------
\section{Metodologia}
\label{sec:introduction:methodology}
%-------------------------------------------------------------

Descrever os passos seguidos para o desenvolvimento da sua pesquisa. Indicar quais foram as técnicas e/ou materiais utilizados na pesquisa, como por exemplo: artigos científicos, livros, entrevistas, ...


%-------------------------------------------------------------
\section{Resumo das Contribuições}
\label{sec:introduction:contributions}
%-------------------------------------------------------------

Breve parágrafo de introdução dessa seção.

\begin{itemize}
  \item Contribuição 1
  \item Contribuição 2
  \item Contribuição 3
  \item Contribuição 4
\end{itemize}


%-------------------------------------------------------------
\section{Estrutura do Documento}
\label{sec:introduction:general-structure}
%-------------------------------------------------------------

Essa dissertação está organizada da seguinte maneira:


\begin{description}

\item[Capítulo~\ref{chp:introduction}.] Compreende essa introdução. A Seção~\ref{sec:introduction:context} descreve o contexto de pesquisa; a Seção~\ref{sec:introduction:motivation}; a Seção ...

\item[Capítulo~\ref{chp:review}.] Proporciona para o leitor uma revisão da literatura técnica e científica relacionadas à pesquisa desenvolvida nessa dissertação. A Seção ... apresenta uma introdução a ... A Seção~\ref{sec:review:related-work}, introduz os trabalhos relacionados identificados ao longo dessa pesquisa.

\item[Capítulo~\ref{chp:approach}.] Apresenta o trabalho desenvolvido ... A Seção ... ; a Seção ...

\item[Capítulo~\ref{chp:experiments}.] Introduz a experimentação ... A Seção ... ; a Seção ... 

\item[Capítulo~\ref{chp:conclusions}.] Apresenta as conclusões chegadas a partir da pesquisa desenvolvida nessa dissertação.

\item[Capítulo~\ref{chp:future-work}.] Comenta as possíveis direções de pesquisa que podem ser desenvolvidas a partir deste trabalho.

\end{description}


%====================================================================
\chapter{Revisão da Literatura}
\label{chp:review}
%====================================================================

\begin{quotation}[Irish dramatist \& poet (1865-1939)]{William B.~Yeats}
    Think like a wise man, \\ but communicate in plain language.
\end{quotation}


\drop Nesse capítulo, você deve dissertar sobre os conceitos e termos envolvidos no contexto e fundamentais para o entendimento do trabalho, referenciando autores da área. Cada seção abaixo, deve abordar um conceito diferente. Você pode subdividir este capítulo distintas seções, conforme julgar necessário.


%-------------------------------------------------------------
\section{Tema 1}
\label{sec:review:topic-1}
%-------------------------------------------------------------

\begin{table}
    \begin{rctabular}{clcl}
        {Icon & Class & Icon & Class}
        X & Y & X & Y \\
        X & Y & X & Y \\ 
        X & Y & X & Y \\
        X & Y & X & Y \\
        X & Y & X & Y \\
    \end{rctabular}
    \caption{Concrete syntax}
    \label{table:concrete-syntax}
\end{table}

Table~\ref{table:concrete-syntax} shows the concrete syntax we use to represent the classes provided by our abstract syntax.

%-------------------------------------------------------------
\section{Tema 2}
\label{sec:review:topic-2}
%-------------------------------------------------------------

%-------------------------------------------------------------
\section{Trabalhos Relacionados}
\label{sec:review:related-work}
%-------------------------------------------------------------

\drop Nesse capítulo você deve descrever e referenciar os trabalhos relacionados encontrados por você na literatura.

De acordo com~\citet{Azanza10}, o mundo é belo, porém a escuridão pode deixá-lo mais triste~\cite{Ballard05,Bergin03}.

%-------------------------------------------------------------
\section{Resumo do Capítulo}
\label{sec:review:summary}
%-------------------------------------------------------------

Nesse capítulo, foi introduzido .... 



%====================================================================
\chapter{Proposta}
\label{chp:approach}
%====================================================================

\begin{quotation}[German Novelist (1749-1832)]{Johann W.~von Goethe}
    Knowing is not enough; we must apply. \\ Willing is not enough; we must do.
\end{quotation}


\drop Nesse capítulo você deve introduzir o trabalho que você desenvolveu.


%-------------------------------------------------------------
\section{Seção 1}
\label{sec:approach:section-1}
%-------------------------------------------------------------

\begin{figure}
	\psfigure{./fig/sample-fig.eps}
	\caption{Esse é um exemplo de como adicionar uma figura.}
	\label{fig:solution}
\end{figure}

A Figura~\ref{fig:solution} representa ...

%-------------------------------------------------------------
\section{Seção 2}
\label{sec:approach:section-2}
%-------------------------------------------------------------

\begin{program}
\begin{listing}[99]
to detectErrors() do \label{line:algorithm-detect-errors:1}
  $q$ = WorkQueue.getInstance() \label{line:algorithm-detect-errors:2}
  repeat \label{line:algorithm-detect-errors:3}
    $b$ = fetch minimum of $q$ (waiting if necessary) \label{line:algorithm-detect-errors:4}
    $c$ = findCorrelation($b$) \label{line:algorithm-detect-errors:5}
    verifyCorrelation($c$) \label{line:algorithm-detect-errors:6}
    for each binding $b$ in $c.nodes$ do \label{line:algorithm-detect-errors:7}
      remove $b$ from $q$ \label{line:algorithm-detect-errors:8}
    end for \label{line:algorithm-detect-errors:9}
  end repeat \label{line:algorithm-detect-errors:10}
end
\end{listing}
    \caption{Algorithm to detect errors}
    \label{fig:algorithm-detect-errors}
\end{program}

In Program~\ref{fig:algorithm-detect-errors}... Line \ref{line:algorithm-detect-errors:1} starts the algorithm....
 
%-------------------------------------------------------------
\section{Seção 3}
\label{sec:approach:section-3}
%-------------------------------------------------------------

A \inline{Process} has to fulfill the following invariants:

\begin{listing}
context Process
    inv: tasks->union(entryPorts.tasks->union(exitPorts.tasks))->
            isUnique(name)
    inv: slots->isUnique(name)
    inv: entryPorts->union(exitPorts)->isUnique(name)
    inv: tasks->select(oclIsKindOf(Communicator))->size() = 0
    inv: let interslots: Set(Slot) = slots->select(s: Slot |
            not self.tasks->includes(s.source) and
            self.tasks->includes(s.target) or
            self.tasks->includes(s.source) and
            not self.tasks->includes(s.target)) in
            interslots->size() = self.entryPorts->size() +
                                 self.exitPorts->size()
\end{listing}

These invariants state that tasks, slots and ports must have unique names, that a process cannot contain any tasks of kind \inline{Communicator} because these tasks are specific to ports, and that there can be only one interslot per port.

%-------------------------------------------------------------
\section{Resumo do Capítulo}
\label{sec:approach:summary}
%-------------------------------------------------------------

Nesse capítulo, foi introduzido .... 





%====================================================================
\chapter{Experimentos}
\label{chp:experiments}
%====================================================================

\begin{quotation}[German Novelist (1749-1832)]{Johann W.~von Goethe}
    Knowing is not enough; we must apply. \\ Willing is not enough; we must do.
\end{quotation}



%-------------------------------------------------------------
\section{Hipóteses}
\label{sec:experiments:hipoteses}
%-------------------------------------------------------------


%-------------------------------------------------------------
\section{Ambiente da Experimentação}
\label{sec:experiments:environment}
%-------------------------------------------------------------


%-------------------------------------------------------------
\section{Variáveis}
\label{sec:experiments:variables}
%-------------------------------------------------------------


%-------------------------------------------------------------
\section{Cenários}
\label{sec:experiments:scenarios}
%-------------------------------------------------------------


%-------------------------------------------------------------
\section{Execução e Coleta de Dados}
\label{sec:experiments:execution}
%-------------------------------------------------------------


%-------------------------------------------------------------
\section{Resultados e Discussão}
\label{sec:experiments:results-discussion}
%-------------------------------------------------------------


%-------------------------------------------------------------
\section{Ameaças à Validade}
\label{sec:experiments:threats-to-validity}
%-------------------------------------------------------------


%-------------------------------------------------------------
\section{Validação}
\label{sec:experiments:validation}
%-------------------------------------------------------------


%-------------------------------------------------------------
\section{Resumo do Capítulo}
\label{sec:experiments:summary}
%-------------------------------------------------------------

Nesse capítulo, foi introduzido .... 

    
%====================================================================
\chapter{Conclusões}
\label{chp:conclusions}
%====================================================================

\begin{quotation}[British author (1564-1616)]{William Shakespeare}
If you can look into the seeds of time, \\
And say which grain will grow and which will not; \\
Speak then to me.
\end{quotation}

\drop Escreva aqui as suas conclusões sobre o trabalho realizado.


%====================================================================
\chapter{Trabalhos Futuros}
\label{chp:future-work}
%====================================================================

\begin{quotation}[Macbeth, Act I, Sc. 3, L. 58]{William Shakespeare}
If you can look into the seeds of time, \\
And say which grain will grow and which will not; \\
Speak then to me.
\end{quotation}

\drop Descreva aqui os trabalhos que podem ser desenvolvidos como continuidade da sua pesquisa. 


\backmatter

% Appendices would be placed here	
	

\makeback{./Bibliography}


\end{document}
