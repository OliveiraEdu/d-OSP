\documentclass{article}
\usepackage{longtable}
\usepackage{multirow}
\usepackage{array}
\usepackage{graphicx}  % For inserting images
\usepackage{caption}   % For better caption formatting
\usepackage{float}     % To control figure placement
\usepackage{natbib}  % Recommended for author-year citations
\usepackage{hyperref}

\title{Dissertation}
\author{Eduardo Oliveira}
\date{\today}

\begin{document}

\maketitle



\section{Blockchain}

\subsection{Origins of Blockchain}

Blockchain is a decentralized, immutable ledger technology designed to facilitate secure and transparent transactions within distributed networks. Initially conceptualized for the Bitcoin blockchain \cite{nakamoto2012bitcoin}, this technology has since evolved into a multi-purpose infrastructure underpinning various domains, including finance, supply chain management, and digital identity verification.

The development of blockchain, however, did not occur in isolation. The concept of a cryptographically secured chain of blocks predates Bitcoin and draws from earlier research on distributed consensus and cryptographic techniques. A key component in blockchain structures is the Merkle tree, introduced by Ralph Merkle in the 1980s \cite{goos_digital_1988}. These trees enable efficient data integrity verification by organizing hashes in a hierarchical structure, which is crucial for maintaining the integrity of blockchain data.

Building on these foundational cryptographic concepts, Stuart Haber and W. Scott Stornetta proposed a method for securely time-stamping digital documents in 1991 \cite{haber_how_1991}. This innovation was significant because it prevented backdating and tampering, laying the groundwork for immutable records. In 1992, Haber, Stornetta, and Bayer further refined this approach by incorporating Merkle trees into their time-stamping system, thereby improving efficiency and strengthening security \cite{bayer_improving_1993}. These advancements not only contributed to the development of blockchain but also highlighted the potential of decentralized, immutable ledgers for maintaining verifiable records.

The inherent properties of blockchain—decentralization, immutability, transparency, and security—make it particularly well-suited for addressing challenges in scientific reproducibility. By maintaining an auditable and tamper-proof history of research data and workflows, blockchain ensures long-term verifiability and integrity. This application leverages the foundational principles established by early cryptographic and distributed systems research, demonstrating how blockchain can extend beyond financial transactions to support rigorous scientific inquiry.

\bibliographystyle{plain}
\bibliography{Bibliography.bib}


\end{document}



