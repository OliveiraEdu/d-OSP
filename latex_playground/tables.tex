\documentclass{article}
\usepackage{longtable}
\usepackage{multirow}
\usepackage{array}
\usepackage{graphicx}  % For inserting images
\usepackage{caption}   % For better caption formatting
\usepackage{float}     % To control figure placement
\usepackage{natbib}  % Recommended for author-year citations
\usepackage{hyperref}
\usepackage{amsmath}
\usepackage{tabularx}
\usepackage{todonotes}
\usepackage{lscape}
\usepackage{geometry}

\geometry{margin=1in}


\title{Literature Review}
\author{Eduardo Oliveira}
\date{\today}

\begin{document}

\maketitle

\listoftodos


\begin{landscape}

    \renewcommand{\arraystretch}{1.4}
    \begin{tabularx}{\linewidth}{|p{3.5cm}|p{2.5cm}|p{3cm}|p{1.8cm}|X|X|X|X|X|}
        \hline
        \textbf{Article Title}                                                                                                             & \textbf{DOI}                    & \textbf{Source}                                                               & \textbf{Publication Date} & \textbf{Main Topics}                                                                                                        & \textbf{Key Takeaways}                                                                                                                         & \textbf{Specific Examples/Use Cases}                                                                    & \textbf{Key Challenges/Opportunities}                                                                                               & \textbf{Overall Perspective}                                                                          \\
        \hline

        An overview of the NFAIS conference: Blockchain for scholarly publishing                                                           & 10.3233/ISU-180015              & Information Services and Use                                                  & 2018                      & NFAIS conference overview; impact on researcher workflows; peer review, IP, research output                                 & Blockchain promises structured, decentralized, secure approach; initiatives exploring use across research lifecycle                            & ARTiFACTS, Po-et, Knowbella Tech, decentralized citation ledgers                                        & Opportunity for horizontal discovery, trust \& transparency; need for awareness \& adoption                                         & Significant long-term potential, short-term expectations might be inflated                            \\
        \hline

        Blockchain and scholarly publishing could be best friends                                                                          & 10.3233/ISU-180016              & Information Services \& Use                                                   & 2018                      & Decentralization, unbundling, creator empowerment; dominance of internet platforms; researcher recognition                  & Blockchain can redistribute power in content discovery, foster trust; focus on creator needs crucial                                           & Steem, BAT, LBRY                                                                                        & Content accessibility \& monetization challenges; opportunity for efficient economic ecosystem; shift in revenue models needed      & Potential for efficient ecosystem, but new revenue models might be required                           \\
        \hline

        Making the unconventional conventional: How blockchain contributes to reshaping scholarly communications                           & 10.3233/ISU-190053              & Information Services \& Use                                                   & 2019                      & Advancing Garfield's vision; blockchain for platforms, open science, recognition, infrastructure; ARTiFACTS focus           & Blockchain can help researchers get credit for all work; ARTiFACTS secures provenance \& attribution                                           & ARTiFACTS platform                                                                                      & Opportunity to make pre-published research accessible, enhance careers                                                              & Blockchain, via platforms like ARTiFACTS, can realize comprehensive researcher recognition            \\
        \hline

        The blockchain and its potential for science and academic publishing                                                               & 10.3233/ISU-180003              & Information Services \& Use                                                   & 2018                      & Peer review, reproducibility, metrics challenges; crypto for funding; DRM; decentralized storage                            & Blockchain could facilitate micropayments, improve rights, create decentralized datastores, enhance metrics                                    & ``Bitcoin for science," Scienceroot, Pluto                                                              & Resistance due to legacy systems \& culture; success depends on implementation level                                                & Potential benefits for scholarly communication, but adoption might be challenging                     \\
        \hline

        A Blockchain Cloud Computing Middleware for Academic Manuscript Submission                                                         & 10.37394/23207.2022.19.51       & WSEAS Transactions on Business and Economics                                  & 2022                      & Improving manuscript submission \& peer review; anonymity, decentralization; reducing bias                                  & Cloud middleware using blockchain can enhance anonymity; aims to optimize peer review                                                          & Four-tier middleware architecture; reviewer selection algorithm; Java Spring \& Ethereum implementation & Opportunity for privacy-focused, decentralized system; challenges in real-world implementation \& scalability                       & Promising results for improving efficiency \& anonymity of manuscript submission \& review            \\
        \hline

        A Review on Blockchain Technology and Blockchain Projects Fostering Open Science                                                   & 10.3389/fbloc.2019.00016        & Frontiers in Blockchain                                                       & 2019                      & Blockchain potential for open science; comparison with open science needs; review of projects                               & Blockchain features align with open science; numerous projects for reproducibility, resource sharing, IP protection                            & Review of 60 blockchain projects                                                                        & Risks \& validation of smart contracts; lack of standardization; sustainable incentives; improving metrics                          & Significant potential for open science infrastructure, but challenges \& community acceptance crucial \\
        \hline

        Ants-Review: A Privacy-Oriented Protocol for Incentivized Open Peer Reviews on Ethereum                                            & 10.1007/978-3-030-71593-9\_2    & Euro-Par 2020: Parallel Processing Workshops                                  & 2021                      & Lack of peer review incentives; blockchain-based system on Ethereum; Ants-Review protocol; tokenomics; community evaluation & Blockchain can reward peer reviewers; Ants-Review offers bounties for anonymous reviews; community voting can promote quality                  & Ants-Review protocol with modules for access, privacy (AZTEC), token management; ANTS token             & Opportunity to improve timeliness, quality, fairness; potential for DeFi \& DAO integration                                         & Promising solution to incentivize peer review, build trust, enhance publication process               \\
        \hline

        Blockchain Adoption in Academia: Promises and Challenges                                                                           & 10.3390/joitmc6040117           & Journal of Open Innovation: Technology, Market, and Complexity                & 2020                      & Promises \& challenges in academia; open data, publishing, funding, governance; adoption barriers                           & Blockchain offers potential for transparency, efficiency, new governance; challenges in usability, security, legal, values                     & Potential applications in open data, peer review, funding via crypto, decentralized communities         & Usability \& security issues; legal concerns; conflict of values; distrust of decentralized governance; need for universal adoption & While promising, significant challenges need addressing for widespread adoption                       \\
        \hline

        Blockchain for scholarly journal evaluation: Potential and prospects                                                               & 10.1002/leap.1408               & Learned Publishing                                                            & 2021                      & Potential for evaluating scholarly journals; enhancing transparency \& reliability of metrics                               & Blockchain could offer more transparent \& tamper-proof way to track \& verify journal performance metrics                                     & Not explicitly detailed                                                                                 & Opportunity to improve credibility of journal evaluation                                                                            & Potential to enhance journal evaluation by providing reliable \& transparent system                   \\
        \hline

        Blockchain solutions for scientific paper peer review: a systematic mapping of the literature                                      & 10.1108/dta-01-2022-0010        & Data Technologies and Applications                                            & 2024                      & Systematic review of blockchain \& smart contracts in peer review; analysis of existing solutions                           & Current solutions focus on incentivizing reviewers; trends include open reviews, Ethereum, publishing ecosystems, IPFS                         & Analysis of 26 articles based on various dimensions                                                     & Opportunity to improve integrity, transparency, efficiency; helps new adopters                                                      & Provides overview of current state, aids future development                                           \\
        \hline

        Fostering Open Data Using Blockchain Technology                                                                                    & 10.1007/978-3-030-77417-2\_16   & Data and Information in Online Environments                                   & 2021                      & Challenges in sharing open data; how blockchain can address authorship, integrity, curation, reputation                     & Blockchain can enable data openness while maintaining ownership; used for certification \& integrity                                           & INPTDAT platform, QPTDat project; use cases for data certification via hash values                      & Balancing data privacy with open sharing; managing on-chain vs. off-chain data                                                      & Promising solutions for fostering open data sharing by addressing key concerns                        \\
        \hline

        Non-Fungible Token (NFT) in the academia and open access publishing environment: Considerations towards science-friendly scenarios & 10.3998/jep.2574                & The Journal of Electronic Publishing                                          & 2022                      & Use \& value of NFTs in academia \& open access; limitations \& disadvantages; science-friendly implementation              & NFTs could restore unique ownership \& value; science-friendly scenarios prioritize cost-free generation, researcher control, interoperability & Scenarios for NFT integration via university presses, submission platforms, DOI agencies                & Reliance on perceived value; fraud potential; gas fees; environmental concerns; interoperability complexities                       & Potential to add value, but careful consideration of science-specific needs necessary                 \\
        \hline

        Open Lab: A web application for running and sharing online experiments                                                             & 10.3758/s13428-021-01776-2      & Behavior Research Methods                                                     & 2022                      & Introduction of Open Lab for deploying \& sharing online experiments; facilitating open science                             & Open Lab simplifies deployment \& sharing, promotes open science by enabling sharing of methods \& data                                        & Open Lab platform features \& integration with OSF                                                      & Lowering technical barriers for online research; fostering collaboration                                                            & User-friendly, open-source solution promoting open science                                            \\
        \hline

        Open-Pub: A Transparent yet Privacy-Preserving Academic Publication System based on Blockchain                                     & 10.1109/icccn52240.2021.9522316 & 2021 International Conference on Computer Communications and Networks (ICCCN) & 2021                      & Proposal of Open-Pub for transparency \& privacy; private Ethereum \& IPFS use                                              & Open-Pub aims to address traditional publishing limits via confidentiality \& transparency; uses tokens for reviewer rewards                   & Architecture of Open-Pub; use of Ethereum \& IPFS; token economy for review process                     & Balancing privacy with transparency; technical complexity; need for adoption in real-world publishing                               & An innovative proposal addressing privacy and transparency in scholarly communication                 \\
        \hline
    \end{tabularx}
\end{landscape}


\begin{table}[ht]
    \centering
    \caption{Recurring Themes and Blockchain Applications in Scholarly Communication}
    \begin{tabularx}{\textwidth}{|X|X|X|X|}
        \hline
        \textbf{Recurring Theme}           & \textbf{Corresponding Blockchain Applications (Examples)}                                        & \textbf{Potential Benefits}                                                                                          & \textbf{Key Challenges}                                                                                     \\
        \hline
        Peer Review Enhancement            & Ants-Review (incentivized reviews), Open-Pub (transparent \& private system)                     & Increased efficiency, transparency, quality, and incentivization of reviewers                                        & Ensuring anonymity, preventing bias, achieving broad adoption among researchers                             \\
        \hline
        Open Science and Data Sharing      & QPTDat project (data certification), Open Lab (sharing experiment methods \& data)               & Improved data integrity, provenance, accessibility, and reproducibility of research                                  & Balancing data privacy with openness, ensuring data quality and curation                                    \\
        \hline
        Author Recognition and Attribution & ARTiFACTS platform (provenance \& attribution), Token system (validated record of contributions) & More comprehensive and validated recognition for diverse academic work, increased control over intellectual property & Establishing meaningful value for non-monetary tokens, ensuring broad acceptance of new recognition metrics \\
        \hline
        Decentralization and Transparency  & Open-Pub (decentralized publication system), Decentralized citation ledgers (NFAIS conference)   & Reduced reliance on intermediaries, increased openness and accountability in publishing processes                    & Overcoming resistance from established institutions, ensuring effective governance of decentralized systems \\
        \hline
        Emerging Applications              & NFTs for digital assets, ``Bitcoin for science'' (concept for research funding)                  & New economic models for scholarly outputs, alternative funding mechanisms for research                               & Addressing environmental concerns of some blockchains, ensuring practical and scalable solutions            \\
        \hline
    \end{tabularx}
\end{table}



\bibliographystyle{plain}
\bibliography{Bibliography.bib}


\end{document}


