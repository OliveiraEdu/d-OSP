
\cite{lawlor_overview_2018} provided an early synthesis of the 2018 NFAIS conference on blockchain in scholarly publishing, emphasizing its potential to reshape research workflows, peer review, and intellectual property management. The article discussed several pilot initiatives—such as ARTiFACTS and Knowbella Tech—that exemplify blockchain’s promise for enhancing provenance tracking and decentralizing research funding. While the enthusiasm for blockchain's role in creating secure, transparent, and decentralized scholarly infrastructures was evident, the article also noted that broad adoption hinges on increased awareness and a more nuanced understanding of the technology's capabilities..

\cite{holmen_blockchain_2018} explores how blockchain could help decentralize scholarly publishing by unbundling traditional workflows and empowering content creators through more equitable recognition and compensation mechanisms. By drawing attention to token-based platforms like Steem, BAT, and LBRY, the article illustrates how blockchain could challenge the dominance of centralized content discovery systems and offer direct value to researchers. A central argument is that focusing on creators' needs—particularly in terms of discoverability and monetization—could lead to more effective publishing ecosystems. While optimistic about blockchain’s potential to reduce delivery-chain costs and enhance transparency, the article also recognizes that such a shift would likely require rethinking current revenue models.

\cite{kochalko_making_2019} presents blockchain as a transformative tool capable of advancing Eugene Garfield’s vision of comprehensive recognition for all research contributions, especially those outside traditional publishing channels. Emphasizing platforms like ARTiFACTS, the article illustrates how blockchain can secure the provenance and attribution of scholarly outputs from the earliest stages of the research lifecycle. By enabling researchers to establish formal recognition for pre-published work, blockchain technologies promise to reshape the incentive structures in academia and elevate the visibility of diverse forms of scholarly activity. The piece conveys a long-term perspective in which blockchain becomes a foundational infrastructure for open science and academic reputation systems.

\cite{van_rossum_blockchain_2018} presents a broad and forward-looking exploration of how blockchain technology could address systemic inefficiencies and credibility issues within science and scholarly publishing. The article underscores blockchain’s potential to reform core academic processes such as peer review and reproducibility, while also opening novel avenues for research funding through cryptocurrency-based incentives. It introduces concepts like decentralized repositories and micropayment systems that could transform access, attribution, and reward mechanisms in academia. Notably, the article frames blockchain as an enabler of decentralized digital rights management and improved research metrics, capable of supporting a more transparent and equitable scholarly ecosystem. However, the author also notes that meaningful adoption will require overcoming significant inertia embedded in existing institutional frameworks and cultural practices.

\cite{gazis_blockchain_2022} in this article proposes a blockchain-based cloud middleware framework aimed at improving academic manuscript submission and peer-review processes. Rather than suggesting a complete overhaul of existing systems, the authors emphasize integration with current infrastructures to enhance anonymity, reduce bias, and increase decentralization. The paper presents a four-tier middleware architecture and a reviewer selection algorithm, both designed to optimize the peer-review workflow. Utilizing open-source tools such as Java Spring and the Ethereum blockchain, the proposed framework demonstrates potential for creating a more privacy-focused and decentralized submission system. Preliminary testing with simulated data supports the framework’s effectiveness, although the authors acknowledge challenges related to real-world scalability and implementation.

\cite{leible_review_2019} in this article offers a systematic review of blockchain’s application to open science, highlighting the alignment between blockchain’s core features—decentralization, immutability, and transparency—and the principles of open science. Through the categorization of 60 blockchain-based projects, the article provides a comprehensive overview of the field as it stood in 2019, identifying trends such as efforts to improve reproducibility, enable secure resource sharing, and support intellectual property protection. The authors also compare blockchain’s technical characteristics with the infrastructural needs of open science ecosystems, emphasizing both the promise and the complexity of this integration. Key challenges identified include the absence of standardization, the technical risks of deploying smart contracts, and the difficulty of designing sustainable incentive models. While the review concludes that blockchain holds considerable promise as a foundation for open science, it stresses that realizing this potential requires addressing these limitations and achieving broader acceptance across the scientific community.

\cite{trovo_ants-review_2021} published as a chapter in Euro-Par 2020: Parallel Processing Workshops (2021), this work addresses the persistent lack of incentives for scientists to engage in peer review by introducing Ants-Review, a blockchain-based protocol designed to support anonymous yet rewarded evaluations using smart contracts on the Ethereum platform (Ants-Review, 2021). The system allows authors to offer bounties in a native token, ANTS, for reviews that satisfy predefined quality standards. To address privacy concerns—critical to wider academic adoption—the protocol incorporates the AZTEC Protocol to ensure anonymity. A gamified community mechanism enables broader participation in evaluating and voting on the quality of peer reviews, thereby promoting ethical behavior and inclusivity. The authors describe core components such as access control, tokenomics, and privacy enforcement, and envision future integrations with Decentralized Finance (DeFi) services and governance through a Decentralized Autonomous Organization (DAO), offering a comprehensive approach to enhancing fairness and efficiency in scientific publishing.

\cite{kosmarski_blockchain_2020} A 2020 article in the Journal of Open Innovation explores how blockchain could enhance open data sharing, peer review, research funding, and governance in academia, while identifying key adoption barriers such as usability challenges, legal uncertainty, and value conflicts between decentralization and academic norms (10.3390/joitmc6040117). In a 2021 Learned Publishing article, the authors examine the potential of blockchain to improve scholarly journal evaluation by providing a transparent, tamper-proof system for tracking metrics like citations and usage, which could increase trust in journal credibility.

\cite{putnings_non-fungible_2022} explores the potential of NFTs to restore a sense of unique ownership and value to digital scholarly works. It proposes science-friendly integration pathways through university presses, submission platforms, and DOI agencies, while emphasizing essential requirements such as cost-free generation and transfer, researcher autonomy, interoperability, and low technical complexity. In a complementary vein, the article “Open Lab: A web application for running and sharing online experiments” (10.3758/s13428-021-01776-2), published in *Behavior Research Methods* (2022), introduces *Open Lab*, a browser-based platform for deploying and sharing experiments created with lab.js. The platform simplifies online research by offering tools for participant management, flexible randomization, and data export, while integrating with the Open Science Framework to support transparency, reproducibility, and open collaboration across the research community.

\cite{zhou_open-pub_2021} this introduces Open-Pub, a blockchain-based academic publishing system designed to balance the seemingly conflicting goals of transparency and privacy. The authors critique current academic publishing models for their limitations in openness, integrity, and data control, proposing a private Ethereum-based blockchain as a viable alternative. A central innovation is the use of a threshold identity-based group signature scheme, which enables secure, anonymous peer review while still ensuring accountability. The system also leverages the InterPlanetary File System (IPFS) for decentralized, persistent storage of both manuscripts and review reports, thereby enhancing accessibility and data integrity. Additionally, *Open-Pub* introduces a token-based incentive mechanism to reward reviewers, reinforcing community engagement and improving the peer review process. Although the paper does not elaborate extensively on incentives for authors, it outlines a comprehensive architecture and feature set aimed at fostering a more equitable and privacy-conscious scholarly communication model. The development of *Open-Pub* exemplifies the growing interest in blockchain technologies to reimagine academic publishing infrastructures, particularly in contexts requiring both control and openness.

\cite{lee_unblocking_2023} critiques the inadequacies of current academic recognition systems, which often privilege publications and citations while neglecting other vital academic contributions. It proposes a blockchain-backed token system that rewards a broader range of academic activities—such as peer review, committee participation, and report submission—with non-tradable, non-monetisable tokens. These tokens serve as a transparent and validated record of contributions, aiming to enhance professional assessments and incentivize engagement in essential but frequently overlooked academic tasks. Designed within the context of the National Institute for Health and Care Research (NIHR), the system is positioned to improve recognition for academic service, increase efficiency in peer review and reporting, and foster cross-funder collaboration. While highlighting significant benefits, the article also acknowledges potential challenges, including public skepticism of blockchain, data protection concerns, regulatory compliance, and the need to ensure token value within academic cultures. The article fits into broader trends across scholarly publishing literature, where blockchain is increasingly seen as a means to improve peer review, incentivize data sharing, enable better author attribution, and support the decentralization and transparency of publishing processes. Themes such as the use of NFTs for digital academic assets, token systems for incentivization, and blockchain-backed models for open science are prominent, alongside debates over usability, governance, legal frameworks, and the role of traditional publishers. The shift from conceptual models to more concrete implementations and critical evaluations reflects a maturing research area that is moving beyond hype to consider the practical application and integration of blockchain into scholarly communication.

\todo{Remove The 2021 book chapter *Fostering Open Data Using Blockchain Technology* explores how blockchain can address key challenges in open data sharing by providing secure proof of authorship, ensuring data integrity, and supporting reputation systems for researchers (10.1007/978-3-030-77417-2_16). It highlights blockchain’s potential to verify data provenance without exposing raw data and discusses approaches like off-chain data management and smart contracts to balance openness with privacy.}


\todo{Discussion and Future Directions}

Blockchain technology possesses a unique combination of characteristics that offer significant potential to address several long-standing challenges within the scholarly publishing ecosystem. Its immutable ledger system has the capacity to enhance transparency and trust across various critical processes, including peer review, the management of research data, and the allocation of research funding, thereby fostering greater confidence among all stakeholders involved.2 The integration of smart contracts can automate numerous currently manual and time-consuming processes, such as the distribution of rewards for peer reviewers, the disbursement of research grants, and the management of intellectual property rights, potentially leading to significant increases in overall efficiency.4 Furthermore, the development of blockchain-based token systems and decentralized platforms offers innovative ways to provide more comprehensive and validated recognition for the diverse contributions that academics make to the scholarly community, extending beyond traditional publication metrics.2 The potential for increased accessibility to scholarly content is another key benefit, with blockchain-based open access platforms potentially reducing the financial barriers to accessing research findings.2 Finally, blockchain's inherent ability to ensure the integrity and provenance of research data can contribute significantly to improving the overall reproducibility of scientific findings, a critical concern within the academic world.4 Blockchain's unique combination of transparency, immutability, and automation capabilities positions it as a powerful tool for tackling core challenges in scholarly publishing. These features directly address long-standing issues related to trust, efficiency, and fairness within the academic communication ecosystem.
Despite the considerable potential, the widespread and successful adoption of blockchain technology in scholarly publishing is contingent upon addressing several practical considerations and overcoming potential hurdles. The scalability and performance of current blockchain networks need to be carefully evaluated to ensure they can effectively handle the high volume of transactions that are characteristic of the scholarly publishing industry. Ensuring usability and facilitating user adoption are also paramount; user-friendly interfaces and seamless integration with researchers' existing workflows will be crucial for widespread acceptance.10 The establishment of clear and effective governance models and standardization protocols for blockchain applications within scholarly publishing is a necessary step to ensure interoperability and build confidence in the technology.10 Interoperability between different blockchain platforms and the existing publishing infrastructure will also be essential for a cohesive and functional ecosystem.45 The cost of developing and maintaining blockchain-based systems, as well as their long-term sustainability from both a financial and environmental perspective, are critical factors that need careful consideration.1 Finally, clarifying the legal and regulatory issues surrounding data ownership, intellectual property rights, and the use of blockchain technology in academic contexts will be essential for fostering a secure and legally sound environment for its adoption.10 Overcoming the practical hurdles related to scalability, usability, governance, and cost will be critical for the successful implementation of blockchain in scholarly publishing. While the theoretical benefits are compelling, the practical challenges of integrating a new technology into an established ecosystem need careful consideration and effective solutions.
Looking towards the future, several potential research directions and emerging trends in the application of blockchain to scholarly publishing warrant attention. Further development and rigorous evaluation of specific blockchain-based platforms and protocols designed to enhance peer review processes, promote open science initiatives, and improve author recognition mechanisms are crucial next steps.7 The exploration of the use of Non-Fungible Tokens (NFTs) for managing and potentially creating new economic models around various forms of scholarly outputs, extending beyond traditional publications, presents an exciting avenue for future research.45 Investigating the potential for integrating blockchain technology with other emerging technologies, such as Artificial Intelligence (AI), to further enhance the efficiency and effectiveness of scholarly communication is another promising direction. Research into the impact of decentralized autonomous organizations (DAOs) on the governance of scholarly publishing initiatives could also yield valuable insights.7 Finally, conducting comprehensive studies on the broader economic and social implications of widespread blockchain adoption within the academic sphere will be essential for a thorough understanding of its long-term impact. Future research should focus on developing practical solutions, addressing implementation challenges, and exploring the synergies between blockchain and other emerging technologies to fully realize its potential in scholarly publishing. Building on the foundational research, the next phase should involve concrete development, rigorous evaluation, and a broader understanding of the wider implications of blockchain adoption.



\todo{Start revision here}


\subsection{Decentralized Technologies: A Novel Frontier for Reproducible and Open Science}

Decentralized technologies, most notably blockchain, are emerging as a novel and potentially transformative frontier in the quest to enhance reproducibility and transparency in scientific research. Blockchain technology, with its inherent characteristics of immutability, transparency, and decentralized operation, offers unique capabilities that could address some of the fundamental challenges related to data integrity, provenance tracking, and the secure sharing of research outputs. Once data is recorded on a blockchain, it becomes virtually impossible to alter or delete, creating a permanent and auditable record of scientific information. This immutability strengthens the trustworthiness of research data and findings. The transparent nature of blockchain allows all participants within the network to verify the information stored, fostering greater accountability and openness in the scientific process.

Complementing blockchain, the InterPlanetary File System (IPFS) presents another promising decentralized technology for open science, offering a distributed peer-to-peer network designed for storing and sharing data. Unlike traditional centralized storage systems, IPFS employs content addressing, where files are uniquely identified based on their content rather than their location. This content-addressed system ensures data integrity, as any alteration to a file would result in a different content identifier (CID), making tampering easily detectable. IPFS facilitates efficient data retrieval by allowing users to access files from multiple nodes simultaneously, improving accessibility and redundancy. Its decentralized nature reduces the problem of data silos associated with central servers, fostering a more open and collaborative environment for scientific data sharing. Projects like the IPFS pinning service for open climate research data demonstrate its suitability for managing and sharing large scientific datasets in a manner that aligns with the FAIR principles (Findable, Accessible, Interoperable, and Reusable) of open science.

Smart contracts, which are self-executing agreements written in code and stored on a blockchain, represent a further application of decentralized technology with significant potential for transforming scientific research. These contracts can automate various aspects of the research process, such as managing access permissions to research data, enforcing the terms of research agreements, and potentially even streamlining peer review and funding mechanisms in a transparent and verifiable way. For example, a researcher could use a smart contract to specify the conditions under which others can access and utilize their data, including any fees, permitted uses, and requirements for attribution. Smart contracts can also be employed to automate the management of encryption keys for sensitive research data, ensuring secure access and preventing unauthorized breaches. By encoding the rules and conditions directly into the blockchain, smart contracts enable trusted transactions and agreements between disparate parties without the need for a centralized authority, potentially increasing the efficiency and transparency of scientific collaborations and data sharing.



\subsection{Synergistic Solutions: Platforms Integrating Open Science and Decentralized Technologies}

The convergence of Open Science principles with the capabilities of decentralized technologies has spurred the emergence of the Decentralized Science (DeSci) movement. This growing movement represents a collaborative and decentralized approach to science, leveraging technological advancements such as Distributed Ledger Technology (DLT), Web3, cryptocurrencies, and Decentralized Autonomous Organizations (DAOs) to foster a more permissionless, open, and inclusive scientific ecosystem.

DeSci initiatives are exploring innovative ways to address the limitations of traditional scientific systems, including issues related to funding, publishing, peer review, and the sharing of research data, by harnessing the unique features of blockchain and related technologies. For example, DAOs are being utilized to facilitate community-driven funding for research projects, promoting high-risk, high-reward investigations that might be overlooked by traditional funding bodies.

The core aim of DeSci is to create a better incentive system for scientists by being open, transparent, and enforced through blockchain technology. Several platforms are actively integrating blockchain and IPFS to establish decentralized infrastructure for open access publishing and the sharing of research data. These platforms strive to eliminate the paywalls associated with traditional academic publishing, ensure the integrity and immutability of research records through blockchain, and provide researchers with greater autonomy and control over their intellectual property.

ResearchHub, for instance, describes itself as a "GitHub for science" and operates as a DAO-governed platform that incentivizes publishing, peer review, and discussion by rewarding contributors with RSC tokens. DeSci Labs is another blockchain-powered platform focused on creating open-science infrastructure for publishing and collaboration, featuring DeSci Publish for sharing manuscripts, datasets, and code while ensuring data immutability and decentralized ownership. ScieNFT utilizes NFT technology to mint and tokenize research outputs, thereby ensuring verifiable ownership and creating new funding mechanisms for research, with storage often managed through IPFS and Filecoin.

Various projects are also exploring the synergistic use of blockchain and IPFS for secure and decentralized file sharing, particularly for research data, with blockchain managing access control and IPFS providing scalable and resilient storage. Smart contracts are being increasingly incorporated into these decentralized open science platforms to automate a range of functionalities, adding layers of transparency, efficiency, and trust to the research workflow. For example, smart contracts can be used to manage intellectual property rights, such as through the tokenization of research outputs as IP-NFTs, which encode the terms of ownership and transfer. They can also facilitate and incentivize peer review processes in a transparent and verifiable manner, potentially rewarding reviewers with tokens for their contributions. A significant application lies in controlling access to research data stored on decentralized platforms like IPFS, where smart contracts can define and enforce access policies based on user roles, permissions, or even conditions like patient consent in the case of medical data. This automation of data access management through smart contracts ensures that data is only shared with authorized parties under predefined and transparent conditions, enhancing both security and privacy.

Table 4: Examples of Decentralized Open Science Platforms and Projects

\subsection{Navigating the Landscape: Limitations and Challenges in Open Science Implementation}

Despite the considerable promise of Open Science practices in addressing the reproducibility crisis, their widespread adoption is not without significant limitations and challenges. Financial barriers pose a substantial hurdle, particularly for researchers in early career stages, those lacking job security, or those affiliated with institutions that have limited financial resources.

The costs associated with gold open-access publishing, where authors or their institutions pay article processing charges (APCs) to make their work immediately and freely available, can be prohibitive. While waivers and institutional funds may exist, disparities persist, especially for researchers from developing nations or smaller teaching institutions.

Social barriers also play a crucial role in hindering adoption. Concerns about career progression, where publishing in fully open access venues might not be perceived with the same prestige as in traditional high-impact factor journals by senior scientists, can deter junior researchers. The fear of potential retaliation for providing critical feedback in open peer review, particularly for those in more junior or precarious positions, also presents a challenge. Furthermore, the significant time investment required for practices like thorough data documentation, archiving, and preregistration, coupled with a lack of strong incentives within the current academic reward system, can make it difficult for researchers already under pressure to "publish or perish" to fully embrace open science.

Implementing Open Science at a large scale also presents considerable challenges related to data governance and technical infrastructure. Ensuring the privacy and security of research data, especially when dealing with sensitive information, requires careful consideration and robust safeguards in an open environment. The need for scalability and consistency across diverse research domains, each with its own unique data types, standards, and practices, poses a significant challenge to the widespread implementation of open science.

Achieving a high degree of automation in open science workflows, from data management to analysis and dissemination, is crucial for efficiency but requires the development and adoption of standardized protocols and interoperable tools. The lack of formal education and training procedures for teaching open science practices also contributes to the challenges in its implementation.

While decentralized technologies offer promising avenues for enhancing open science and reproducibility, they are not without their own limitations and challenges. User experience and the learning curve associated with adopting new blockchain-based or IPFS-based platforms can be significant barriers, particularly for researchers who are not technologically inclined. Potential performance issues, such as slower data retrieval times on decentralized networks compared to centralized servers, and scalability limitations as the volume of data and users grows, need to be addressed.

Security concerns, including the potential for malicious content on decentralized storage networks and vulnerabilities in smart contract code, also require careful mitigation strategies. Ensuring data availability on decentralized storage systems like IPFS can be challenging if content is not widely replicated. Furthermore, the lack of robust governance mechanisms and standardization across different decentralized platforms could hinder interoperability and widespread adoption within the scientific community.

The costs associated with implementing and maintaining blockchain infrastructure, as well as the computational power required for certain operations, can also be considerable.

\subsection{Evaluating the Impact: Effectiveness of Approaches and the Role of Decentralized Technologies}

A growing body of evidence suggests that the adoption of Open Science practices is indeed having a positive impact on the reproducibility and reliability of scientific research. Studies have begun to demonstrate a correlation between practices such as open data sharing and higher rates of replication success. For instance, a replication study in the field of Artificial Intelligence revealed a strong positive relationship between the sharing of both code and data and the ability to reproduce research findings. The principle of making research accessible and usable through open research and open data has been shown to accelerate scientific discovery and strengthen the reliability of results.

Furthermore, the increased transparency and collaboration fostered by Open Science contribute to enhanced reproducibility and overall trust in research. The practice of preregistering study protocols has also been linked to improvements in research accuracy, potentially by reducing bias in reporting outcomes. These findings indicate that the fundamental principles and practices of Open Science are playing a crucial role in addressing the reproducibility crisis and fostering a more robust and trustworthy scientific ecosystem.

Initial evaluations of the application of decentralized technologies like blockchain and IPFS in scientific research point towards their potential to significantly enhance transparency, data integrity, and accessibility, which are all critical components of research reproducibility. Blockchain technology, with its inherent immutability and transparency, has the capacity to strengthen the verification process in science, potentially leading to more reproducible and reliable research results. The use of blockchain in Decentralized Science (DeSci) initiatives aims to ensure data immutability and incentivize practices like reproducibility and peer review.

Moreover, blockchain can facilitate the verification of authoritative data and the tracking of scientific resource sharing, thereby promoting open science and reproducibility. IPFS, as a decentralized storage solution, enhances data availability and integrity by distributing data across a network and using content-based addressing. While these initial evaluations are promising, the field is still relatively nascent, and further research is needed to comprehensively assess the long-term impact of these technologies on research reproducibility and to effectively address the limitations and challenges associated with their implementation.

Examining case studies and examples of platforms and projects that have successfully integrated Open Science principles with decentralized technologies provides valuable insights into the practical applications and tangible benefits of these approaches. Platforms like ResearchHub are leveraging blockchain technology to create open access publishing environments that incentivize collaboration and reward contributions. VitaDAO exemplifies a decentralized autonomous organization that funds longevity research through community-driven governance. Projects utilizing IPFS and smart contracts are demonstrating the feasibility of secure and decentralized file sharing for research data, with blockchain managing access control and IPFS providing resilient storage. The development of blockchain-based dynamic consent platforms for clinical trials showcases the potential of these technologies to enhance transparency and data integrity in sensitive research areas. These real-world examples highlight the diverse ways in which the integration of Open Science and decentralized technologies can lead to more open, transparent, and potentially more reproducible scientific practices across various research domains.

\subsection{Conclusion: Charting a Course for a Reproducible and Open Scientific Ecosystem}

This literature review has explored the critical issues surrounding the reproducibility crisis in scientific research, the emergence and core tenets of the Open Science movement as a response, and the potential of decentralized technologies to further enhance reproducibility and openness.

The analysis indicates that the reproducibility crisis is a multifaceted problem stemming from systemic issues within the publication system, questionable research practices, and challenges related to statistical rigor and reporting. This crisis has significant implications for the credibility of scientific knowledge, public trust in science, and the efficient allocation of research resources.

The Open Science movement, with its principles of transparency, accessibility, collaboration, and reproducibility, offers a promising pathway towards mitigating the reproducibility crisis. Practices such as open data and methods, preprints and open access publishing, preregistration and registered reports, and open peer review have demonstrated the potential to increase the rigor and reliability of scientific findings. Furthermore, current initiatives from funding agencies, academic institutions, and the development of various tools and platforms are actively promoting the adoption of more reproducible research practices within the scientific community.

Decentralized technologies, particularly blockchain, IPFS, and smart contracts, represent a novel and potentially transformative frontier for advancing open and reproducible science. Blockchain's immutability and transparency can enhance data integrity and provenance tracking, while IPFS offers a resilient and decentralized infrastructure for storing and sharing research data. Smart contracts can automate key processes such as data access control and the enforcement of research agreements, adding layers of transparency and efficiency. The emergence of Decentralized Science (DeSci) initiatives signifies a growing effort to integrate these technologies with Open Science principles, leading to the development of innovative platforms for decentralized funding, publishing, and data sharing. Despite the significant potential of Open Science and decentralized technologies, their widespread implementation faces several challenges.

Financial and social barriers can hinder the adoption of open practices, while concerns related to data privacy, security, scalability, consistency, and automation need to be carefully addressed. Decentralized technologies also come with their own set of limitations, including usability issues, performance variability, and the need for robust governance and standardization.Growing evidence suggests that Open Science practices are indeed improving research reproducibility. Initial evaluations of decentralized technologies in science also indicate their potential to enhance transparency and data integrity. Case studies and examples of integrated platforms demonstrate the feasibility and benefits of combining these approaches in various research contexts.

Moving forward, future research should focus on further evaluating the effectiveness of different Open Science practices and the impact of decentralized technologies on research reproducibility across various disciplines. Addressing the limitations and challenges associated with their implementation, particularly regarding scalability, user experience, and governance, will be crucial.

Recommendations for researchers include embracing open science practices and exploring the responsible use of decentralized technologies in their work. Institutions and funding agencies should continue to develop policies and provide resources that support open and reproducible research. Technology developers should focus on creating user-friendly, scalable, and secure decentralized platforms tailored to the needs of the scientific community. By collaboratively addressing these issues, the scientific community can chart a course towards a more reproducible and open scientific ecosystem that fosters trust, accelerates discovery, and benefits society as a whole.


\begin{table}[h]
    \centering
    \caption{Reproducibility Failure Rates Across Scientific Disciplines}
    \begin{tabular}{|l|c|}
        \hline
        \textbf{Scientific Field}   & \textbf{Failure Rate (\%)} \\
        \hline
        Chemistry                   & 87                         \\
        Biology                     & 77                         \\
        Medicine                    & 67                         \\
        Top-tier Journal Studies    & 78                         \\
        Preclinical Cancer Research & 89                         \\
        \hline
    \end{tabular}
    \label{tab:reproducibility}
\end{table}








\subsection{NIH Office of Data Science Strategy (ODSS)}

The National Institutes of Health (NIH) established the Office of Data Science Strategy (ODSS) to coordinate and advance efforts in biomedical data science across NIH institutes. ODSS supports the development of a FAIR-compliant infrastructure that improves data discoverability, access, and reuse. Key programs include the STRIDES Initiative, which expands cloud access for biomedical researchers, and efforts related to the NIH Data Commons Pilot, aimed at integrating data from disparate sources. These activities underscore NIH’s leadership in implementing scalable and sustainable Research Data Management (RDM) frameworks that support reproducibility, transparency, and collaborative innovation in biomedical research \cite{odss_nih}.

\subsection{NSF Open Knowledge Network (OKN)}

The National Science Foundation (NSF) launched the Open Knowledge Network (OKN) to foster a national-scale semantic infrastructure that enables the linking and integration of heterogeneous datasets. The OKN initiative promotes the creation of interconnected knowledge graphs to facilitate discovery, contextualization, and machine readability of research outputs across disciplines. Its overarching goal is to transform how data is shared and reused within and across scientific communities by supporting novel data integration architectures and semantic modeling techniques \cite{nsf_okn}. As a key investment in the U.S. Open Science landscape, the OKN advances interoperability and supports evidence-based decision-making through rich, machine-actionable knowledge representations.

\subsection{OSTP “Nelson Memo” (2022)}

In 2022, the White House Office of Science and Technology Policy (OSTP) issued a directive known as the “Nelson Memo,” which requires that all federally funded research publications and associated data be made immediately and freely available to the public by December 31, 2025. This policy marks a pivotal shift in U.S. open access strategy by eliminating embargo periods and strengthening mandates for data transparency. Building on previous open science policies, the Nelson Memo seeks to ensure equitable access to publicly funded knowledge, drive reproducibility, and accelerate scientific progress through a national commitment to openness and accountability \cite{ostp_nelson}.

\subsection{RDA-US (U.S. Research Data Alliance Node)}

RDA-US serves as the American node of the international Research Data Alliance, contributing to the development of technical and social frameworks that support global research data sharing. Through its national outreach efforts, RDA-US connects U.S.-based researchers and institutions to a wider global network dedicated to interoperable data infrastructures. It plays an active role in facilitating community-led working groups, promoting standards adoption, and informing policy dialogues related to data stewardship and open science. As part of the broader RDA ecosystem, RDA-US reinforces the alignment of domestic RDM practices with evolving international norms and best practices \cite{rda_us}.


\begin{table}[H]
    \centering
    \caption{Comparison of Open Science and RDM Initiatives}
    \label{tab:initiative_comparison}
    \begin{tabularx}{\textwidth}{|X|X|X|X|}
        \hline
        \textbf{Initiative} & \textbf{Scope \& Focus}                          & \textbf{Objectives}                                                  & \textbf{Connection to Open Science and RDM}                                      \\
        \hline
        LEARN Toolkit       & Institutional / Europe (global applicability)    & Provide best practices and guidance for implementing RDM policies    & Supports institutional RDM policies and FAIR-aligned infrastructure              \\
        \hline
        FAIRsFAIR           & European Union / Research Infrastructure         & Foster adoption of FAIR data principles in research                  & Embeds FAIR principles into RDM workflows and European infrastructures           \\
        \hline
        EOSC                & Pan-European / Federated Ecosystem               & Build an open science cloud for data sharing and reuse               & Offers infrastructure for FAIR and open data across disciplines and nations      \\
        \hline
        GO FAIR             & Global / Community-driven Implementation         & Bottom-up operationalization of FAIR principles                      & Promotes FAIR-aligned standards and tools through implementation networks        \\
        \hline
        RDA                 & Global / Multi-sectoral Coordination             & Enable open data sharing through social and technical bridges        & Develops global interoperability standards and community-led guidelines          \\
        \hline
        CODATA              & Global / UNESCO-aligned Policy                   & Improve data quality and accessibility for science and policy        & Coordinates global data policy and supports open science frameworks              \\
        \hline
        OpenAIRE            & European Union / Federated Repositories          & Support Open Access and Open Science via infrastructure              & Connects repositories, supports metadata harvesting, and research graph building \\
        \hline
        NIH ODSS            & United States / Biomedical Research              & Modernize data infrastructure and promote FAIR biomedical research   & Facilitates large-scale data sharing and cross-institutional RDM standards       \\
        \hline
        NSF OKN             & United States / National Research Infrastructure & Build a linked data platform to enhance knowledge discovery          & Advances interoperable data structures and semantic metadata practices           \\
        \hline
        OSTP Nelson Memo    & United States / Federal Policy                   & Mandate immediate public access to federally funded research outputs & Shapes U.S. policy landscape for Open Access and data sharing by 2026            \\
        \hline
        RDA-US              & United States / National Node of RDA             & Coordinate U.S. involvement in global RDA                            & Aligns U.S. data practices with international open data standards and policies   \\
        \hline
    \end{tabularx}
\end{table}




Influenced by the ethos of the Open Source software movement and driven by dissatisfaction with the limitations of traditional scientific publishing outlets, Open Science gained initial visibility through the proliferation of Open Access journals offering unrestricted access to research articles \cite{laakso_anatomy_2012}. This early momentum evolved into a more comprehensive initiative encompassing all phases of the research lifecycle.

For example, a lack of raw data has been identified as a critical issue in many cases where reproducibility failed. Furthermore, complex workflows in emerging fields like big data exacerbate these challenges by introducing technical hurdles such as inadequate computational resources and inconsistent reporting standards. Addressing these issues requires a cultural shift toward open science practices, including mandatory data sharing policies, standardized metadata frameworks, and improved infrastructure for storing and accessing research materials. By prioritizing robust data management systems, the scientific community can mitigate the reproducibility crisis and restore confidence in research integrity.


\begin{table}[h]
    \centering
    \caption{Scope of the Reproducibility Crisis in Science}
    \label{tab:reproducibility_scope}
    \begin{tabular}{|p{5cm}|p{4cm}|p{4cm}|}
        \hline
        \textbf{Scientific Discipline} & \textbf{Reported Reproducibility/Replication Rate} & \textbf{Source}                 \\
        \hline
        General Science                & $>70\%$ failure rate in replication attempts       & \cite{baker2016reproducibility} \\
        Top-Tier Journal Publications  & 22\% replicable                                    & \cite{prinz2011believe}         \\
        General Science                & 15\% high reproducibility                          & \cite{baker2016reproducibility} \\
        Biomedical Research            & 10-40\% reproducible                               & \cite{freedman2015economics}    \\
        High-Profile Experiments       & Up to 70\% replication failure                     & \cite{baker2016reproducibility} \\
        Preclinical Cancer Research    & $\sim11\%$ successfully replicated                 & \cite{prinz2011believe}         \\
        Chemistry                      & 87\% failure to reproduce                          & \cite{baker2016reproducibility} \\
        Biology                        & 77\% failure to reproduce                          & \cite{baker2016reproducibility} \\
        Medicine                       & 67\% failure to reproduce                          & \cite{baker2016reproducibility} \\
        \hline
    \end{tabular}
\end{table}


\subsection{Open Science Practices as Solutions to the Reproducibility Crisis}

Open Science proposes a set of interrelated practices designed to confront the reproducibility crisis by fostering greater transparency, accessibility, and collaboration in scientific inquiry. Among these practices, five core principles stand out: Open Data, Open Materials, Open Access, Preregistration, and Open Analysis. These principles address systemic issues that undermine the credibility and reliability of scientific outputs and seek to realign research practices with the foundational values of openness and verifiability.

A central element of this framework is the commitment to \textbf{Open Data}, which calls for unrestricted access to raw research data and associated metadata. This principle directly addresses the lack of transparency that often impedes reproducibility by ensuring that the empirical foundation of research is available for validation, reinterpretation, and reuse. Open Data repositories serve a critical role in this ecosystem by preserving datasets in standardized formats, maintaining provenance metadata, and enabling persistent access. Provenance—information about the origin, context, and transformations applied to the data—is particularly important, as it supports reproducibility by providing a traceable record of how datasets were collected, processed, and interpreted \cite{burgelman_open_2019}. Without these metadata standards and traceability mechanisms, shared data risk becoming uninterpretable or misleading when repurposed.

Closely linked to Open Data is the principle of \textbf{Open Materials}, which involves making the physical and digital research components—such as experimental protocols, instruments, survey instruments, and software—available for reuse. Open Materials ensure that researchers seeking to replicate a study or extend its methodology have access to the same inputs and tools used in the original work. Depositing these materials in domain-specific repositories and documenting them with clear metadata and provenance records enhances both transparency and usability. Together, Open Data and Open Materials form the empirical and procedural foundation upon which reproducible science is built.

\textbf{Open Access} complements these practices by addressing the dissemination of research outputs. It entails making peer-reviewed publications freely available without subscription or payment barriers. Open Access expands the reach and impact of scientific knowledge, enabling researchers from under-resourced institutions and disciplines to participate in scholarly discourse and replication efforts. In conjunction with preprints—versions of manuscripts shared prior to peer review—Open Access accelerates the circulation of ideas and allows the broader community to scrutinize findings earlier in the research lifecycle. This early-stage visibility invites broader feedback and can help identify methodological flaws or inconsistencies that might otherwise go unnoticed until post-publication.

To strengthen methodological transparency, Open Science also promotes \textbf{Preregistration} and \textbf{Registered Reports}. Preregistration involves submitting a time-stamped protocol outlining the research questions, hypotheses, and planned analyses before data collection begins. This distinction between confirmatory and exploratory research discourages practices such as HARKing (Hypothesizing After Results are Known) and p-hacking. Registered Reports build on this model by subjecting research protocols to peer review before data collection, granting in-principle acceptance based on methodological rigor rather than the nature of the results. These practices align incentives toward robust design and transparent reporting, reducing publication bias and enhancing the credibility of scientific claims.

Finally, \textbf{Open Analysis} entails sharing the code and computational workflows used in data processing and statistical inference. By making analysis pipelines available, researchers allow others to reproduce exact outputs from shared data, supporting both validation and reuse. Integration with containerization tools, version control systems, and computational notebooks strengthens this principle, enabling complete provenance tracking of computational environments and decisions.

Together, these five principles constitute a framework for addressing the epistemic and procedural shortcomings at the heart of the reproducibility crisis. Through open repositories, standardized metadata, and transparent workflows, Open Science reconfigures the production and dissemination of knowledge to support a more trustworthy and collaborative scientific enterprise.
